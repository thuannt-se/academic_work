\section{Preliminaries}
\subsection{Networks and Node, Community Structure}

\indent In mathematics, graph theory is the study of graphs, which are mathematical structures used to model pairwise relations between objects. A graph (or network) in this context is made up of vertices (also called nodes or points) which are connected by edges (also called links or lines).

A node is a unit of a network that represents an entity, such as a person, a product, a group, or a concept. Edges represent relations between nodes. A graph can be considered as the simplest representation of a complex system, where the vertices are the elementary units of the system and the edges represent their mutual interactions.

The term ”community structure” refers to the composition of a given network, including the number and attributes of nodes, as well as the relationships between them. Community structure is a crucial aspect of complex networks, as it indicates that nodes tend to cluster together into distinct groups or communities.

The term ”community structure” refers to the composition of a given network, including the number and attributes of nodes, as well as the relationships between them. Community structure is a crucial aspect of complex networks, as it indicates that nodes tend to cluster together into distinct groups or communities.

Understanding the community structure of a network can have important implications for many real-world applications, such as predicting the spread of diseases or identifying influential individuals in a social network. Moreover, community detection can also help to reveal underlying patterns and structures in complex systems, providing insights into the organization and behavior of these systems.



Comprehending the community structure of a network carries significant implications for various real-world applications, including forecasting the dissemination of diseases and pinpointing influential individuals within a social network. Furthermore, the process of community detection serves to unveil latent patterns and structures within intricate systems, affording insights into the organizational dynamics and behavioral tendencies inherent in these systems.
\subsection{Community Detection problem}
\indent Communities can be implicit or explicit. Explicit communities are those, in which a grouping is predefined and members joining the group form a community. In this case communities are directly visible, for example whatsapp group. 

\indent Implicit communities on the other hand do not have any predefined classification. We have to analyze the activities of the individuals to form the community. Community detection is used for implicit communities only

\indent Community detection is a fundamental problem in network analysis that aims to identify groups of nodes that are more similar to each other than to the rest of the network. 

\indent One can argue that community detection is similar to clustering. Clustering is a machine learning technique in which similar data points are grouped into the same cluster based on their attributes. Even though clustering can be applied to networks, it is a broader field in unsupervised machine learning which deals with multiple attribute types. On the other hand, community detection is specially tailored for network analysis which depends on a single attribute type called edges. Also, clustering algorithms have a tendency to separate single peripheral nodes from the communities it should belong to. However, both clustering and community detection techniques can be applied to many network analysis problems and may raise different pros and cons depending on the domain.

\indent Community detection is widely used in various fields such as biology, sociology, and marketing to understand social structures, reveal user data within the network, and develop relevant recommendation systems. There are several approaches to community detection, including hierarchical clustering, spectral clustering, modularity maximization, and random walk methods. However, they can be broadly categorized into two types; Agglomerative Methods and Divisive Methods.

\indent In Agglomerative methods, edges are added one by one to a graph which only contains nodes. Edges are added from the stronger edge to the weaker edge. Divisive methods follow the opposite of agglomerative methods. In there, edges are removed one by one from a complete graph.

\indent Ever since the discovery of community structure in real-world networks, a plethora of techniques devoted to their detection has been introduced. The challenge is both theoretical, in proposing a good mathematical definition of what constitutes a community, and computational, in developing good heuristics that can detect communities in a reasonable time.
\subsection{Partitioning quality evaluation}
A common way of investigating the community structure of networks starts with the definition of a quality function, which assigns a score to any network partition. Larger scores correspond to better partitions, and algorithms are created to find the partition with the largest score.

A good community partition of a network should have fewer edges between communities than one would expect by chance. This indicates significant community structure within the network. On the other hand, if the number of edges between groups is what would be expected by chance, it is not evidence of significant community structure. The measure known as modularity quantifies this idea that true community structure corresponds to a statistically surprising arrangement of edges, with fewer edges between groups and more edges within groups than expected by chance.
\subsubsection{Modularity}
Modularity Q is one of the most used measure. The modularity Q is, up to a multiplicative constant, the number of edges falling within groups minus the expected number in an equivalent network with edges placed at random. The modularity can be either positive or negative, with positive values indicating the possible presence of community structure. Thus, one can search for community structure precisely by looking for the divisions of a network that have positive, and preferably large, values of the modularity

The original idea of modularity was given by Newman and Girvan, they have defined modularity Q as:\\
$$Q=\frac{1}{2}\sum_{ij}(A_{ij}-\frac{k_{i}k_{j}}{2m})\delta(\sigma_{i},\sigma{j})$$
Here, m is the number of links, $k_{i}$ is the degree of vertex i, $k_{j}$ is the degree of vertex j, $\sigma_{i}$ is the community to vertex i, $\sigma_{j}$ is the community to vertex j, and $\delta(\sigma_{i}\sigma_{j}) = 1$ if i and j belong to the same community, otherwise it equals to 0.\\
An alternate formulation of this is as a sum over communities:
$$Q=\frac{1}{2m}\sum_{c}(A_{c}-\frac{K_{c}^{2}}{4m})$$
where $m_{c}$ is the number of internal edges (or tatal internal edge weight) of community $c$ and $K_{c} = \sum_{i|\sigma_{i}=c}k_{i}$ is the total (weighted) degree of nodes in community $c$.

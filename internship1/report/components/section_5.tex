\section{Conclusion}
\subsection{Achievement}

The algorithms that the group used to solve the community search problem for the artist dataset above accomplished the following:

\begin{itemize}
    \item Our team have implemented and runned 3 algorithms (Girvan Newman, Louvain, Leiden) on Python, from that, we can see some comparisions and how they works.

    \item Using the above data clusters, we can suggest similar books to the user while they are seeing the information of the books they want to buy, increasing the diversity of the e-commerce website such as Tiki.

    \item Analyzing the tastes of the readers community in specific clusters to promote better suggestions and apply them to run ads or PR is another strength that this algorithm brings.

    \item Besides, we also build a website to demonstrate our work and its application. The demonstration video of the website is here: \url{https://youtu.be/nnhuGfggjb4}
\end{itemize}

\subsection{Drawback}
Because of the limited research time, the group did not generate many algorithms or make very specific comparisons with the algorithms sought. \\
However, with the above three algorithms,Girvan-Newmana and Louvain, classical algorithms, and Leiden, an improved algorithm widely used today, the group has also made appropriate comparisons of performance and reliability to generalize them.

\subsection{Future work}
The team plans to use more algorithms in the future to get a thorough understanding of the approaches used by businesses and researchers to address problems.

The team also intends to work with more actual data sets in order to fully assess the algorithm’s efficacy and its potential for use in other contexts. This deals with, for instance, developing techniques for providing suggestions or advertising using these data clusters.
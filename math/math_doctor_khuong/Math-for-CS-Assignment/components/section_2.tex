\section{Preliminaries}
\subsection{Networks and Node, Community Structure}
A node is a unit of a network that represents an entity, such as a person, a group, or a concept. A network is a collection of nodes connected by edges or links, which can represent various types of relationships between the nodes. Networks are a common way of representing complex systems in a wide variety of scientific fields, from social sciences to biology, networks often exhibit a natural division into clusters or modules called communities. At the heart of these communities are nodes, which represent individual entities within the network. These nodes typically have unique attributes and connections to other nodes within the same community. A network is a collection of nodes, also called vertices, that are connected by edges or links, representing the relationships between them. The existence of community structure implies that vertices within a community share certain commonalities and exhibit similar behaviors within the larger graph. \\

For example, in a transportation network, nodes could represent cities or transportation hubs, with edges representing the links between them such as roads, railways, or airlines. Within this network, communities could emerge based on geographical proximity or shared transportation infrastructure. For example, cities located along the same highway or railway line may form a community, as they share similar transportation options and patterns of movement. Similarly, airlines may form communities based on shared routes or alliances, as they tend to have more connections and interactions with each other than with other airlines outside their community. \\

The term "community structure" refers to the composition of a given network, including the number and attributes of nodes, as well as the relationships between them. Community structure is a crucial aspect of complex networks, as it indicates that nodes tend to cluster together into distinct groups or communities. Within each community, the density of edges is typically higher than between communities, emphasizing the strength of connections among members of the same community.\\ 

Understanding the community structure of a network can have important implications for many real-world applications, such as predicting the spread of diseases or identifying influential individuals in a social network. Moreover, community detection can also help to reveal underlying patterns and structures in complex systems, providing insights into the organization and behavior of these systems.
\subsection{Community Detection}
Community detection is a fundamental problem in network analysis that aims to identify groups of nodes that are more similar to each other than to the rest of the network. Community detection is widely used in various fields such as biology, sociology, and marketing to understand social structures, reveal user data within the network, and develop relevant recommendation systems.
There are several approaches to community detection, including hierarchical clustering, spectral clustering, modularity maximization, and random walk methods. Each method has its strengths and weaknesses, and the choice of method depends on the specific properties of the network being analyzed.
\begin{itemize}
    \item Modularity maximization is one of the most widely used methods for community detection, and it aims to partition the network into non-overlapping communities that maximize a quality function called modularity. Modularity measures the density of edges within communities relative to the expected density if edges were distributed randomly. Maximal modularity indicates strong community structure, while low modularity suggests a random network.
    \item Hierarchical clustering is a method that groups nodes based on their similarity and generates a hierarchical tree of communities. In contrast, spectral clustering uses eigenvectors of the graph Laplacian to cluster nodes and has been shown to perform well on networks with well-defined community structures.
    \item The Girvan-Newman algorithm is a popular method for community detection in networks. The algorithm works by iteratively removing edges from the network, with the aim of breaking it into smaller components that correspond to communities.
\end{itemize}
Community detection has many applications, including identifying functional modules in biological networks, understanding the structure of social networks, and detecting communities in web graphs. In recent years, there has been growing interest in developing community detection algorithms that are scalable, efficient, and applicable to large-scale networks.
\subsection{Modularity}
A good community partition of a network should have fewer edges between communities than one would expect by chance. This indicates significant community structure within the network. On the other hand, if the number of edges between groups is what would be expected by chance, it is not evidence of significant community structure. The measure known as modularity quantifies this idea that true community structure corresponds to a statistically surprising arrangement of edges, with fewer edges between groups and more edges within groups than expected by chance.\\
\subsubsection{Modularity Density}
Li et al. have developed a new quantitative metric called modularity density (D), which is based on the density of subgraphs, to determine community structure of networks.\\
The higher the value of D, the better is a partition. The optimization of modularity density is also NP-hard. The modularity density is defined as:\\
$$D_\lambda=\sum_{i=1}^{m} \frac{2\lambda L(V_i, V_i)-2(1-\lambda)L(V_i-\overline{V1})}{\vert {V_i}\vert }$$
where, $V_i$  is the subset of V i=1,..., m, such that $L(V_i)=\sum_{i\in V_{i}, j\in \overline{V_1}}A_{ij}$, where  $\overline{V_1}=VV_i$
\subsubsection{Modularity Q}
Modularity Q is one of the most used measure. The modularity Q is, up to a multiplicative constant, the number of edges falling within groups minus the expected number in an equivalent network with edges placed at random.
The modularity can be either positive or negative, with positive values indicating the possible presence of community structure. Thus, one can search for community structure precisely by looking for the divisions of a network that have positive, and preferably large, values of the modularity.\\
The original idea of modularity was given by Newman and Girvan, they have defined modularity Q as:\\
$$Q=\frac{1}{2}\sum_{ij}(A_{ij}-\frac{k_{i}k_{j}}{2m})\delta(\sigma_{i},\sigma{j})$$
Here, m is the number of links, $k_{i}$ is the degree of vertex i, $k_{j}$ is the degree of vertex j, $\sigma_{i}$ is the community to vertex i, $\sigma_{j}$ is the community to vertex j, and $\delta(\sigma_{i}\sigma_{j}) = 1$ if i and j belong to the same community, otherwise it equals to 0.\\
An alternate formulation of this is as a sum over communities:
$$Q=\frac{1}{2m}\sum_{c}(A_{c}-\frac{K_{c}^{2}}{4m})$$
where $m_{c}$ is the number of internal edges (or tatal internal edge weight) of community $c$ and $K_{c} = \sum_{i|\sigma_{i}=c}k_{i}$ is the total (weighted) degree of nodes in community $c$.

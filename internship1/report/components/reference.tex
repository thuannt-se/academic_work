\newpage
\section*{References.}
\renewcommand\refname{}
\addcontentsline{toc}{section}{\protect\numberline{}References}%


\begin{thebibliography}{00}


% \bibitem{data-driven-business}
% He, Lifeng, et al. "A linear-time two-scan labeling algorithm." Proceedings of the 2007 IEEE International Conference on Image Processing (ICIP), vol. 5, IEEE, 2007

% \bibitem{data-driven-business}
% Connected-component labeling. Truy cập ngày 15/04/2023\\
% \url{https://en.wikipedia.org/wiki/Connected-component_labeling#One_component_at_a_time}


% \bibitem{data-driven-business}
% The connected-component labeling problem: A review of state-of-the-art algorithms. Truy cập ngày 20/04/2023\\
% \url{https://www.sciencedirect.com/science/article/pii/S0031320317301693#bib0061}
% \end{thebibliography}

\bibitem{Dataset}
Dataset source: \url{ https://www.kaggle.com/datasets/arashnic/book-recommendation-dataset}

\bibitem{Supervised Learning reference}
Hastie, T., Tibshirani, R., Friedman, J. (2009). The elements of statistical learning: data mining, inference, and prediction. Springer Science Business Media.\\

\bibitem{Unsupervised Learning}
Bishop, C. M. (2006). Pattern recognition and machine learning. Springer.

\bibitem{Semi-supervised Learning}
Zhu, X., Goldberg, A. B., Nowak, R. (2009). Introduction to semi-supervised learning. Synthesis Lectures on Artificial Intelligence and Machine Learning, 3(1), 1-130.\\

\bibitem{Reinforcement learning}
Sutton, R. S.,  Barto, A. G. (2018). Reinforcement learning: An introduction. MIT Press. \\

\bibitem{Machine learning applications}
Chen, H., Zhang, J., Hsu, W. (2017). Data-driven optimization for ride-sharing systems: A reinforcement learning approach. Transportation Research Part C: Emerging Technologies, 80, 48-63.\\
\bibitem{Machine learning applications}
Li, Y., Zhu, J., Gong, Y. (2018). Deep reinforcement learning for person re-identification in video surveillance. In Proceedings of the European Conference on Computer Vision (ECCV) Workshops (pp. 0-0).\\
\bibitem{Machine learning applications}
Pang, C., Li, Z. (2018). Deep learning based large scale visual recommendation and search for E-commerce. In Proceedings of the Twenty-Seventh International Joint Conference on Artificial Intelligence (IJCAI) (pp. 0-0).\\

\bibitem{Machine learning applications}
Domingos, P. (2012). A few useful things to know about machine learning. Communications of the ACM, 55(10), 78-87.

\bibitem{IoT}
Atzori, L., Iera, A., Morabito, G. (2010). The Internet of Things: A survey. Computer Networks, 54(15), 2787-2805.

\bibitem{IoT}
Gubbi, J., Buyya, R., Marusic, S., Palaniswami, M. (2013). Internet of Things (IoT): A vision, architectural elements, and future directions. Future Generation Computer Systems, 29(7), 1645-1660.

\bibitem{Applications of IoT}
Elgazzar, Khalid and Khalil, Haytham and Alghamdi, Taghreed and Badr, Ahmed and Abdelkader, Ghadeer and Elewah, Abdelrahman and Buyya, Rajkumar. (2022). Revisiting the internet of things: New trends, opportunities and grand challenges\\
\url{http://dx.doi.org/10.3389/friot.2022.1073780}

\bibitem{Neural Network}
Haykin, S. (1994). Neural networks: a comprehensive foundation. Prentice Hall.

\bibitem{Neural Network}
Hagan, M. T., Demuth, H. B., & Beale, M. H. (2014). Neural network design. PWS Publishing Company.

\bibitem{Neural Network}
Goodfellow, I., Bengio, Y., & Courville, A. (2016). Deep learning. MIT press.

\bibitem{Neural Network}
Janiesch, C., Zschech, P. & Heinrich, K. Machine learning and deep learning. Electron Markets 31, 685–695 (2021).

\bibitem{Neural Network}
Ullah, A.; Anwar, S.M.; Li, J.; Nadeem, L.; Mahmood, T.; Rehman, A.; Saba, T. Smart cities: The role of Internet of Things and machine learning in realizing a data-centric smart environment. Complex Intell. Syst. 2023, 1–31

\bibitem{Neural Network}
Islam, M.R.; Kabir, M.M.; Mridha, M.F.; Alfarhood, S.; Safran, M.; Che, D. Deep Learning-Based IoT System for Remote Monitoring and Early Detection of Health Issues in Real-Time. Sensors 2023, 23, 5204.

\bibitem{Neural Network}
Elhanashi, A.; Dini, P.; Saponara, S.; Zheng, Q. Integration of Deep Learning into the IoT: A Survey of Techniques and Challenges for Real-World Applications. Electronics 2023, 12, 4925.

\bibitem{Edge computing}
W. Yu et al., "A Survey on the Edge Computing for the Internet of Things," in IEEE Access, vol. 6, pp. 6900-6919, 2018, doi: 10.1109/ACCESS.2017.2778504.

\bibitem{Deep learning and Edge computing}
F. Wang, M. Zhang, X. Wang, X. Ma and J. Liu, "Deep Learning for Edge Computing Applications: A State-of-the-Art Survey," in IEEE Access, vol. 8, pp. 58322-58336, 2020, doi: 10.1109/ACCESS.2020.2982411.

\bibitem{Deep learning and Edge computing}
Rong, G., Xu, Y., Tong, X., & Fan, H. (2021). An edge-cloud collaborative computing platform for building AIoT applications efficiently. Journal of Cloud Computing, 10(1), 1-14. 
\url{https://doi.org/10.1186/s13677-021-00250-w}

\bibitem{TinyML and AIoT}
Ray, P. P. (2022). A review on TinyML: State-of-the-art and prospects. Journal of King Saud University - Computer and Information Sciences, 34(4), 1595-1623.
\url{https://doi.org/10.1016/j.jksuci.2021.11.019}

\bibitem{TinyML and AIoT}
Hou, K. M., Diao, X., Shi, H., Ding, H., Zhou, H., & De Vaulx, C. (2022). Trends and Challenges in AIoT/IIoT/IoT Implementation. Sensors, 23(11), 5074.
\url{https://doi.org/10.3390/s23115074}

\bibitem{TinyML and AIoT}
ST. X-CUBE-AI Artificial Intelligence (AI) Software Expansion for STM32Cube; STMicroelectronics: Geneva, Switzerland, 2021;
\url{https://www.st.com/resource/en/data_brief/x-cube-ai.pdf}

\bibitem{TinyML and AIoT}
ST. STM32CubeMX for STM32 Configuration and Initialization C Code Generation. June 2022 UM1718 Rev 38
\url{https://www.st.com/resource/en/data_brief/stm32cubemx.pdf}

\bibitem{TinyML and AIoT}
TinyML talk Felix Johnny and Fredrik Knutsson - Arm Sweden Area Group – February 8, 2021
\url{https://cms.tinyml.org/wp-content/uploads/emea2021/tinyML_Talks_Felix_Johnny_Thomasmathibalan_and_Fredrik_Knutsson_210208.pdf}

\end{thebibliography}
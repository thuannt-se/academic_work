\documentclass{article}
\usepackage[utf8]{inputenc}
\usepackage{amsmath}
\usepackage{amssymb}
\title{Homework - 1}
\author{Thuan Nguyen - Thai Tang - Khuong Dinh - Hung Vo - Long Trieu}
\begin{document}

\maketitle

\section{Exercise 1}

(a) Factor these two matrices into $A = XAX^{-1}$:
\[
A = 
\begin{bmatrix}
    1 & 2 \\
    0 & 3
\end{bmatrix}
\]
We have that: $\det(\lambda I-A) = 0$ \\
$= \begin{bmatrix}
    1 -\lambda & 2 \\
    0 & 3 -\lambda
\end{bmatrix}$ 
$=\lambda^2 - 4\lambda + 3$
\\So there are 2 roots: $\lambda = 1$ or $\lambda = 3$.    
\\For these numbers, the matrix becomes singular (non-determinant). We have
\( (A - \lambda \mathbf{I}) x_1 = 0 \) is \( A x_1 = x_1 \Rightarrow x_1 = \begin{bmatrix}
    1 \\
    0
\end{bmatrix} \)
\\
\( (A - \lambda \mathbf{I}) x_2 = 0 \) is \( A x_2 = 3 x_2 \Rightarrow x_2 = \begin{bmatrix}
    1 \\
    1
\end{bmatrix} \)
\\ 
We have \(x_1\) and \(x_2\) as eigenvectors \(\Rightarrow\) matrix X = \(\begin{bmatrix}
    1  & 1\\
    0 & 1
\end{bmatrix}\)

To calculate \(A = X \Lambda X^{-1}\), we first need to compute \(X^{-1}\), the inverse of matrix \(X\). In this case, \(X\) is invertible, so we can calculate \(X^{-1}\).
\\Let's calculate \( X^{-1} \):
\[
X^{-1} = \begin{bmatrix}
    1 & -1 \\
    0 & 1
\end{bmatrix}
\]

Now, we can substitute the values into the equation \( A = X \Lambda X^{-1} \):
\[
A = \begin{bmatrix}
    1 & 1 \\
    0 & 1
\end{bmatrix}
\begin{bmatrix}
    1 & 2 \\
    0 & 3
\end{bmatrix}
\begin{bmatrix}
    1 & -1 \\
    0 & 1
\end{bmatrix}
\]
Now we can calculate \(A^3 = X \Lambda^3 X^{-1} \) = 
\[
A = \begin{bmatrix}
    1 & 1 \\
    0 & 1
\end{bmatrix}
\begin{bmatrix}
    1 & 0 \\
    0 & 27
\end{bmatrix}
\begin{bmatrix}
    1 & -1 \\
    0 & 1
\end{bmatrix}
\]
= \[ \begin{bmatrix}
    1 & 27 \\
    0 & 27
\end{bmatrix}
\begin{bmatrix}
    1 & -1 \\
    0 & 1
\end{bmatrix}
\]
= \[\begin{bmatrix}
    1 & 26 \\
    0 & 27
\end{bmatrix} \]
\\ Similarly, we can calculate the diagonalized matrix for: 
\[
A = 
\begin{bmatrix}
    1 & 1 \\
    3 & 3
\end{bmatrix}
\]
\\ there are 2 eigenvalues: $\lambda = 0$ or $\lambda = 4$.    
\\ The diagonalize matrix is:
\[
A = 
\begin{bmatrix}
    1 & 1 \\
    3 & 3
\end{bmatrix}
= \begin{bmatrix}
    1 & 1 \\
    -1 & 3
\end{bmatrix}
\begin{bmatrix}
    0 & 0 \\
    0 & 4
\end{bmatrix}
\begin{bmatrix}
    1/4 & -1/4 \\
    1/4 & 1/4
\end{bmatrix}
\]
%%%%%%%%%%%%%%%%%%%%%%%%%%%%%%%%%%%%%%%%%
\section{Exercise 3}

We can replace the fact that \( A = X \Lambda X^{-1} \) and \( I = X X^{-1} \) into \( A + 2I \), we will have:

\[
A + 2I = X \Lambda X^{-1} + 2 X X^{-1}
= X (\Lambda + 2I) X^{-1}
\]

So, the eigenvalue matrix is \( \Lambda + 2I \), the eigenvector matrix is \( X \).
%%%%%%%%%%%%%%%%%%%%%%%%%%%%%%%%%%%%%%%%%
\section{Exercise 4}
(a) False. Even if all eigenvectors of \(A\) are linearly independent, some still can correspond to eigenvalue \(\lambda = 0\). Because the determinant of the matrix \(A\) is the product of its eigenvalues, we have \(\det(A) = 0\), which implies that \(A\) cannot be invertible.
\\
(b) True. Because we have \(n\) independent eigenvectors that can form a basis \(\mathbb{R}^n\) for matrix \(A \in \mathbb{R}^{n \times n}\), it follows that \(A\) is diagonalizable.
\\
(c) True. Because all the columns of \(X\) are independent, \(X\) is a full-rank matrix, which implies that \(X\) can be inverted.
\\
(d) False. Not enough to draw a conclusion, because there are invertible matrices can not be diagonalize.
%%%%%%%%%%%%%%%%%%%%%%%%%%%%%%%%%%%%%%%%%
\section{Exercise 6}
\[
A=
  \begin{bmatrix}
    4 & 0 \\
    1 & 2 
  \end{bmatrix}
\]
$\det(\lambda I-A) = 0$ \\
$\Rightarrow \det\left( 
\begin{bmatrix}
    1 & 0 \\
    0 & 1 
\end{bmatrix} -
\begin{bmatrix}
    4 & 0 \\
    1 & 2 
\end{bmatrix}\right) = 
\begin{bmatrix}
    \lambda - 4  & 0 \\
    0 & \lambda -2 
\end{bmatrix} = 0.$
So there are two roots: \\
$\lambda = 2$ or $\lambda = 4$.    
\\
\\
Find eigenvectors for $\lambda = 4$. Find all vectors $X \neq 0$ such that $AX = 4X$.
\\
We have:
$(4I - A) X = 0 \Rightarrow X =   
\begin{bmatrix}
    4 & 0 \\
    1 & 0 
\end{bmatrix}
\begin{bmatrix}
    x  \\
    y  
\end{bmatrix} = \begin{bmatrix}
    0  \\
    0  
\end{bmatrix} \Rightarrow y = 0 \Rightarrow x = \begin{bmatrix}
2i  \\
i  
\end{bmatrix}$
\\
\\
Find eigenvectors for $\lambda = 2$. Find all vectors $X \neq 0$ such that $AX = 2X$.
\\
We have:
$(2I - A) X = 0 \Rightarrow X =   
\begin{bmatrix}
    2 & 0 \\
    1 & 0 
\end{bmatrix}
\begin{bmatrix}
    x  \\
    y  
\end{bmatrix} = \begin{bmatrix}
    0  \\
    0  
\end{bmatrix} \Rightarrow y = 0 \Rightarrow x = \begin{bmatrix}
i  \\
0  
\end{bmatrix}$
\\
$\Rightarrow S = \begin{bmatrix}
2 & 0  \\
1 & 1  
\end{bmatrix}$.
Having that $A = X \cap X^{-1} = X^{-1} \cap (X^{-1})^{-1}.$
\\ We can conclude that all the matrices are diagonal. $A^{-1}$ are the inverse matrices $X^{-1}$ of matrix $X$.
%%%%%%%%%%%%%%%%%%%%%%%%%%%%%%%%%%%%
\section{Exercise 12}
(a) False. Multiples of $(1,4)$ eigenvector could respond to a nonzero eigenvalue $\Rightarrow$ $\det(A) \neq 0$ $\Rightarrow$ $A$ is invertible.
\\
(b) True. If not, we would have distinct (independent) eigenvectors.
\\
(c) True, since there are not enough independent eigenvectors $\Rightarrow$ eigenvector matrix $X$ is not invertible $\Rightarrow$ $A$ is not diagonalizable.

\end{document}
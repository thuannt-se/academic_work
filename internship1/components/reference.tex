\newpage
\section*{References.}
\renewcommand\refname{}
\addcontentsline{toc}{section}{\protect\numberline{}References}%


\begin{thebibliography}{00}


% \bibitem{data-driven-business}
% He, Lifeng, et al. "A linear-time two-scan labeling algorithm." Proceedings of the 2007 IEEE International Conference on Image Processing (ICIP), vol. 5, IEEE, 2007

% \bibitem{data-driven-business}
% Connected-component labeling. Truy cập ngày 15/04/2023\\
% \url{https://en.wikipedia.org/wiki/Connected-component_labeling#One_component_at_a_time}


% \bibitem{data-driven-business}
% The connected-component labeling problem: A review of state-of-the-art algorithms. Truy cập ngày 20/04/2023\\
% \url{https://www.sciencedirect.com/science/article/pii/S0031320317301693#bib0061}
% \end{thebibliography}

\bibitem{Dataset}
Dataset source: \url{ https://www.kaggle.com/datasets/arashnic/book-recommendation-dataset}

\bibitem{Supervised Learning reference}
Hastie, T., Tibshirani, R., & Friedman, J. (2009). The elements of statistical learning: data mining, inference, and prediction. Springer Science & Business Media.\\

\bibitem{Unsupervised Learning}
Bishop, C. M. (2006). Pattern recognition and machine learning. Springer.

\bibitem{Semi-supervised Learning}
Zhu, X., Goldberg, A. B., & Nowak, R. (2009). Introduction to semi-supervised learning. Synthesis Lectures on Artificial Intelligence and Machine Learning, 3(1), 1-130.\\

\bibitem{Reinforcement learning}
Sutton, R. S., & Barto, A. G. (2018). Reinforcement learning: An introduction. MIT Press. \\

\bibitem{Machine learning applications}
Chen, H., Zhang, J., & Hsu, W. (2017). Data-driven optimization for ride-sharing systems: A reinforcement learning approach. Transportation Research Part C: Emerging Technologies, 80, 48-63.\\
\bibitem{Machine learning applications}
Li, Y., Zhu, J., & Gong, Y. (2018). Deep reinforcement learning for person re-identification in video surveillance. In Proceedings of the European Conference on Computer Vision (ECCV) Workshops (pp. 0-0).\\
\bibitem{Machine learning applications}
Pang, C., & Li, Z. (2018). Deep learning based large scale visual recommendation and search for E-commerce. In Proceedings of the Twenty-Seventh International Joint Conference on Artificial Intelligence (IJCAI) (pp. 0-0).\\

\bibitem{Machine learning applications}
Domingos, P. (2012). A few useful things to know about machine learning. Communications of the ACM, 55(10), 78-87.

\bibitem{IoT}
Atzori, L., Iera, A., & Morabito, G. (2010). The Internet of Things: A survey. Computer Networks, 54(15), 2787-2805.

\bibitem{IoT}
Gubbi, J., Buyya, R., Marusic, S., & Palaniswami, M. (2013). Internet of Things (IoT): A vision, architectural elements, and future directions. Future Generation Computer Systems, 29(7), 1645-1660.

\bibitem{Applications of IoT}
Elgazzar, Khalid and Khalil, Haytham and Alghamdi, Taghreed and Badr, Ahmed and Abdelkader, Ghadeer and Elewah, Abdelrahman and Buyya, Rajkumar. (2022). Revisiting the internet of things: New trends, opportunities and grand challenges\\
\url{http://dx.doi.org/10.3389/friot.2022.1073780}


\end{thebibliography}
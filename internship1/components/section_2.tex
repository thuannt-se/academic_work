\section{Preliminaries}
\subsection{Networks and Node, Community Structure}

\indent A node, in the context of network theory, serves as a fundamental unit signifying an entity, be it an individual, a collective, or an abstract concept. Networks, comprised of nodes, are interconnected through edges or links, symbolizing diverse relational aspects between these nodes. The representation of intricate systems is a prevalent practice across numerous scientific domains, ranging from the social sciences to biology. Notably, networks frequently manifest a discernible organization into clusters or modules referred to as communities. At the core of these communities lie nodes, emblematic of individual entities within the network, distinguished by unique attributes and interconnections with other nodes within the same community. Described alternatively as vertices, nodes constitute the integral components of a network, and their interlinkages via edges or links elucidate the underlying relationships. The discernible presence of a community structure suggests that vertices within a given community share specific commonalities and display analogous behaviors within the overarching graph. \\

To illustrate, consider a transportation network where nodes serve as representations of cities or transportation hubs, and edges depict the interconnecting links, such as roads, railways, or airlines. In this network, the emergence of communities is conceivable, rooted in factors like geographical proximity or common transportation infrastructure. For instance, cities situated along the same highway or railway line may constitute a community, given their shared transportation options and movement patterns. Likewise, airlines might form communities based on shared routes or alliances, as they exhibit a higher degree of connections and interactions among themselves compared to interactions with airlines outside their respective community.\\

The term "community structure" pertains to the configuration of a specific network, encompassing considerations such as the quantity and characteristics of nodes, along with the nature of their interconnections. In the realm of complex networks, the concept of community structure holds paramount significance, denoting the tendency of nodes to coalesce into discernible groups or communities. Notably, within each community, there is generally a heightened density of edges compared to the connections observed between distinct communities. This disparity underscores the robustness of relationships among members within the same community, thereby delineating the structural cohesion inherent in such network configurations.\\ 

Comprehending the community structure of a network carries significant implications for various real-world applications, including forecasting the dissemination of diseases and pinpointing influential individuals within a social network. Furthermore, the process of community detection serves to unveil latent patterns and structures within intricate systems, affording insights into the organizational dynamics and behavioral tendencies inherent in these systems.
\subsection{Community Detection}
\indent The identification of communities is a foundational challenge in network analysis, seeking to discern sets of nodes characterized by greater similarity among themselves than with the broader network. This problem holds broad applicability across diverse domains, including biology, sociology, and marketing, where it facilitates the comprehension of social structures, the extraction of user-related information within the network, and the formulation of pertinent recommendation systems.\\
\indent Various methodologies are employed for community detection, encompassing hierarchical clustering, spectral clustering, modularity maximization, and random walk techniques. Each approach possesses distinct advantages and limitations, and the selection of a specific method hinges upon the unique properties inherent in the network under examination.
\begin{itemize}
    \item Among the array of community detection methods, modularity maximization stands out as one of the most extensively employed approaches. The primary objective of this method is to divide the network into non-overlapping communities in a manner that maximizes a quality metric known as modularity. Modularity gauges the edge density within communities in contrast to the anticipated density under random edge distribution. Elevated modularity values signify a robust community structure, indicative of a network with well-defined clusters, whereas lower modularity values imply a more randomized network configuration.
    \item Hierarchical clustering is a methodology that organizes nodes according to their similarity, yielding a hierarchical tree representation of communities. In contrast, spectral clustering leverages the eigenvectors of the graph Laplacian to cluster nodes and has demonstrated efficacy particularly in networks characterized by clearly delineated community structures.
    \item The Girvan-Newman algorithm is a widely adopted approach for community detection in networks. This algorithm operates through an iterative process of removing edges from the network, intending to systematically partition it into smaller components that align with distinct communities.
\end{itemize}
Community detection has many applications, including identifying functional modules in biological networks, understanding the structure of social networks, and detecting communities in web graphs. In recent years, there has been growing interest in developing community detection algorithms that are scalable, efficient, and applicable to large-scale networks.
\subsection{Modularity}
A good community partition of a network should have fewer edges between communities than one would expect by chance. This indicates significant community structure within the network. On the other hand, if the number of edges between groups is what would be expected by chance, it is not evidence of significant community structure. The measure known as modularity quantifies this idea that true community structure corresponds to a statistically surprising arrangement of edges, with fewer edges between groups and more edges within groups than expected by chance.\\
\subsubsection{Modularity Density}
Li et al. have introduced a novel quantitative metric known as modularity density (D), rooted in the density of subgraphs, for the purpose of delineating the community structure within networks. A higher value of D is indicative of a more favorable partition. It is noteworthy that the optimization of modularity density is classified as an NP-hard problem, underscoring the computational complexity associated with achieving an optimal partition based on this metric.\\
$$D_\lambda=\sum_{i=1}^{m} \frac{2\lambda L(V_i, V_i)-2(1-\lambda)L(V_i-\overline{V1})}{\vert {V_i}\vert }$$
where, $V_i$  is the subset of V i=1,..., m, such that $L(V_i)=\sum_{i\in V_{i}, j\in \overline{V_1}}A_{ij}$, where  $\overline{V_1}=VV_i$
\subsubsection{Modularity Q}
Modularity (Q) stands out as one of the most commonly employed measures in network analysis. Essentially, modularity Q, up to a multiplicative constant, quantifies the number of edges within groups, contrasting this with the expected number in an equivalent network where edges are randomly distributed. The modularity value can be either positive or negative. Positive values suggest the potential existence of community structure within the network. Consequently, the identification of community structure involves the exploration of network partitions characterized by positive, preferably substantial, modularity values.\\
The original idea of modularity was given by Newman and Girvan, they have defined modularity Q as:\\
$$Q=\frac{1}{2}\sum_{ij}(A_{ij}-\frac{k_{i}k_{j}}{2m})\delta(\sigma_{i},\sigma{j})$$
Here, m is the number of links, $k_{i}$ is the degree of vertex i, $k_{j}$ is the degree of vertex j, $\sigma_{i}$ is the community to vertex i, $\sigma_{j}$ is the community to vertex j, and $\delta(\sigma_{i}\sigma_{j}) = 1$ if i and j belong to the same community, otherwise it equals to 0.\\
An alternate formulation of this is as a sum over communities:
$$Q=\frac{1}{2m}\sum_{c}(A_{c}-\frac{K_{c}^{2}}{4m})$$
where $m_{c}$ is the number of internal edges (or tatal internal edge weight) of community $c$ and $K_{c} = \sum_{i|\sigma_{i}=c}k_{i}$ is the total (weighted) degree of nodes in community $c$.
